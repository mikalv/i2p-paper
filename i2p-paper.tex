\documentclass[a4paper,twocolumn,12pt]{article}
\usepackage{url}

\title{I2P: An anonymous overlay network}

\author{Mikal Villa}

\begin{document}
\maketitle

\begin{abstract}\noindent
This paper will present I2P, a scalable, self organizing, resilient packet switched anonymous network layer, upon which any number of different anonymity or security conscious applications can operate. Each of these applications may make their own anonymity, latency, and throughput tradeoffs without worrying about the proper implementation of a free route mixnet, allowing them to blend their activity with the larger anonymity set of users already running on top of I2P.
\end{abstract}

\tableofcontents

\section{High level overview}

I2P is designed to allow peers using I2P to communicate with each other anonymously - both sender and recipient are unidentifiable to each other as well as to third parties. Having the ability to run servers within I2P is essential, as it is quite likely that any outbound proxies to the normal Internet will be monitored, disabled, or even taken over to attempt more malicious attacks. I2P is not a research project - academic, commercial, or governmental, but is instead an engineering effort aimed at doing whatever is necessary to provide a sufficient level of anonymity to those who need it. But it is encouraged to research on I2P.

The network itself is message oriented - it is essentially a secure and anonymous IP layer, where messages are addressed to cryptographic keys (Destinations) and can be significantly larger than IP packets. Some example uses of the network include "eepsites" (webservers hosting normal web applications within I2P), a BitTorrent client ("I2PSnark"), or a distributed data store. With the help of the I2PTunnel application, we are able to stream traditional TCP/IP applications over I2P. Most people will not use I2P directly, or even need to know they're using it.

To deal with a wide range of attacks, I2P is fully distributed with no centralized resources - and hence there are no directory servers keeping statistics regarding the performance and reliability of routers within the network. As such, each router must keep and maintain profiles of various routers and is responsible for selecting appropriate peers to meet the anonymity, performance, and reliability needs of the users.

The network itself makes use of a significant number of cryptographic techniques and algorithms - a full laundry list includes 2048bit ElGamal encryption, 256bit AES in CBC mode with PKCS\#5 padding, 1024bit DSA signatures, SHA256 hashes, 2048bit Diffie-Hellman negotiated connections with station to station authentication, and ElGamal / AES+SessionTag.

Note that while there have been several simple SOCKS proxies available to tie existing applications into the network, their value has been limited as nearly every application routinely exposes what, in an anonymous context, is sensitive information. The only safe way to go is to fully audit an application to ensure proper operation and to assist in that we provide a series of APIs in various languages which can be used to make the most out of the network.


\subsection{Destinations}

A destination is a 516-character Base64 cryptographic identifier

\subsection{Base32 Names}

I2P supports Base32 hostnames similar to Tor's .onion addresses. Base32 addresses are much shorter and easier to handle than the full 516-character Base64 Destinations or addresshelpers. Example: 
\url{ukeu3k5oycgaauneqgtnvselmt4yemvoilkln7jpvamvfx7dnkdq.b32.i2p}

In Tor, the address is 16 characters (80 bits), or half of the SHA-1 hash.
\cite{tor-hiddenservice}
I2P uses 52 characters (256 bits) to represent the full SHA-256 hash. The form is {52 chars}.b32.i2p. Tor has recently published a proposal to convert to an identical format of {52 chars}.onion for their hidden services. Base32 is implemented in the naming service, which queries the router over I2CP to lookup the LeaseSet to get the full Destination. Base32 lookups will only be successful when the Destination is up and publishing a LeaseSet. Because resolution may require a network database lookup, it may take significantly longer than a local address book lookup.

Base32 addresses can be used in most places where hostnames or full destinations are used, however there are some exceptions where they may fail if the name does not immediately resolve. I2PTunnel will fail, for example, if the name does not resolve to a destination.

\subsection{Tunnels}

Another critical concept to understand is the "tunnel". A tunnel is a directed path through an explicitly selected list of routers. Layered encryption is used, so each of the routers can only decrypt a single layer. The decrypted information contains the IP of the next router, along with the encrypted information to be forwarded. Each tunnel has a starting point (the first router, also known as "gateway") and an end point. Messages can be sent only in one way. To send messages back, another tunnel is required.

Two types of tunnels exist: "outbound" tunnels send messages away from the tunnel creator, while "inbound" tunnels bring messages to the tunnel creator. Combining these two tunnels allows users to send messages to each other. The sender sets up an outbound tunnel, while the receiver creates an inbound tunnel. The gateway of an inbound tunnel can receive messages from any other user and will send them on until the endpoint. The endpoint of the outbound tunnel will need to send the message on to the gateway of the inbound tunnel. To do this, the sender adds instructions to her encrypted message. Once the endpoint of the outbound tunnel decrypts the message, it will have instructions to forward the message to the correct inbound gateway.

\subsection{Network database}

A third critical concept to understand is I2P's "network database" (or "netDb") - a pair of algorithms used to share network metadata. The two types of metadata carried are "routerInfo" and "leaseSets" - the routerInfo gives routers the data necessary for contacting a particular router (their public keys, transport addresses, etc), while the leaseSet gives routers the information necessary for contacting a particular destination. A leaseSet contains a number of "leases". Each of this leases specifies a tunnel gateway, which allows reaching a specific destination.


\bibliographystyle{plain}
\bibliography{references}


\end{document}
